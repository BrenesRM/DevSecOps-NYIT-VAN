\documentclass[conference]{IEEEtran}
\IEEEoverridecommandlockouts
\usepackage{geometry}
\usepackage{hyperref}
\usepackage{url}

\title{Project Proposal: Enhancing Software Supply Chain Security in DevSecOps CI/CD Pipelines}
\author{
    \IEEEauthorblockN{Marlon Brenes R}
    \IEEEauthorblockA{
        Student ID: 1314316 \\
        Master's in Cybersecurity \\
        \href{mailto:marlon.brenes@nyit.edu}{marlon.brenes@nyit.edu}
    }
}
\date{}

\begin{document}

\maketitle

\begin{abstract}
This project aims to design, implement, and demonstrate a secure Software Supply Chain framework within DevSecOps CI/CD pipelines by leveraging GitHub Actions, Snyk, SonarCloud, OWASP standards, ArgoCD, and Kubernetes. The focus is on mitigating supply chain risks and improving software delivery security through automation and best practices.
\end{abstract}

\section{Introduction}
The speed of product development in today's business models requires incorporating security into the software supply chain from the project's inception. High-profile attacks, such as SolarWinds and Log4j, have underscored the critical need for secure processes. Current tools like Snyk and SonarCloud address specific vulnerabilities but often lack seamless integration into DevSecOps processes. This project proposes a cohesive, automated framework that integrates these tools and methodologies, building upon existing work and standards such as NIST SP 800-204D.

\section{Problem Statement}
The goal is to create a secure Software Supply Chain framework that automates security integration into a DevSecOps pipeline. This framework will address the limitations of existing solutions by combining tools like GitHub Actions, Snyk, and SonarCloud with best practices from OWASP and secure deployment mechanisms using ArgoCD and Kubernetes.

\section{Project Objectives}
\begin{itemize}
    \item Develop a secure CI/CD pipeline using GitHub Actions.
    \item Integrate security tools such as Snyk and SonarCloud for automated scanning.
    \item Apply OWASP best practices to the pipeline.
    \item Securely deploy applications using ArgoCD and Kubernetes.
    \item Simulate and mitigate common supply chain misconfiguration scenarios.
    \item Document results, challenges, and recommendations.
\end{itemize}

\section{Methodology}
The project methodology is divided into the following phases:

\subsection{Pipeline Setup}
\begin{itemize}
    \item Create a CI/CD pipeline using GitHub Actions.
    \item Integrate automated tools like Snyk and SonarCloud for code and dependency scanning.
\end{itemize}

\subsection{Security Enhancements}
\begin{itemize}
    \item Implement OWASP best practices, including OWASP ZAP (Zed Attack Proxy).
    \item Configure Kubernetes with role-based access control (RBAC) and network policies.
\end{itemize}

\subsection{Application Deployment}
\begin{itemize}
    \item Use ArgoCD for continuous delivery to Kubernetes clusters.
    \item Ensure secure deployments using signed container images and configurations.
\end{itemize}

\subsection{Testing and Validation}
\begin{itemize}
    \item Simulate common supply chain misconfigurations.
    \item Evaluate the pipeline's resilience and effectiveness.
\end{itemize}

\subsection{Documentation and Analysis}
\begin{itemize}
    \item Record methodology, results, challenges, and improvements.
\end{itemize}

\section{Resources}
\subsection{Tools}
GitHub Actions, Snyk, SonarCloud, OWASP guidelines (e.g., OWASP ZAP), ArgoCD.

\subsection{Software}
Kubernetes (Minikube/AKS), Docker, secure image repositories.

\subsection{Hardware}
Cloud-based or local testing environments (e.g., Azure or virtual machines).

\subsection{Data}
Open-source vulnerable applications (e.g., OWASP Juice Shop).

\section{Project Schedule}
\begin{itemize}
    \item \textbf{Week 1-2}: Research and define project scope; set up GitHub Actions.
    \item \textbf{Week 3-5}: Integrate Snyk and SonarCloud; document initial findings.
    \item \textbf{Week 6-8}: Apply OWASP best practices; set up ArgoCD and Kubernetes.
    \item \textbf{Week 9-11}: Simulate and test supply chain attacks; refine pipeline security.
    \item \textbf{Week 12-14}: Document results; finalize project paper and presentation.
\end{itemize}

\section{Contribution to Knowledge}
This project aims to:
\begin{itemize}
    \item Demonstrate a secure DevSecOps pipeline framework.
    \item Highlight the integration of security tools and practices.
    \item Address automation gaps in supply chain security.
    \item Provide recommendations for secure software delivery processes.
\end{itemize}

\section{References}
\begin{thebibliography}{1}
    \bibitem{owasp} OWASP, "OWASP ZAP (Zed Attack Proxy)." [Online]. Available: \url{https://github.com/zaproxy/action-api-scan}
    \bibitem{snyk} Snyk, "Vulnerability Scanning for Developers." [Online]. Available: \url{https://snyk.io}
    \bibitem{sonarcloud} SonarCloud, "Static and Dynamic Code Analysis Tools." [Online]. Available: \url{https://sonarcloud.io}
    \bibitem{argocd} ArgoCD, "GitOps Continuous Delivery." [Online]. Available: \url{https://argoproj.github.io}
    \bibitem{kubernetes} Kubernetes, "Official Documentation." [Online]. Available: \url{https://kubernetes.io/docs/home/}
    \bibitem{github} GitHub, "Official Repository of the Project." [Online]. Available: \url{https://github.com/BrenesRM/DevSecOps-NYIT-VAN}
\end{thebibliography}

\end{document}
