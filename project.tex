\documentclass[a4paper,12pt]{article}
\usepackage{geometry}
\geometry{top=1in, bottom=1in, left=1in, right=1in}
\usepackage{hyperref}

\title{Project Proposal: Enhancing Software Supply Chain Security in DevSecOps CI/CD Pipelines}
\author{Marlon Brenes R \\
        Student ID: 1314316}
\date{\today}

\begin{document}

\maketitle

\section{Problem Statement}
To design, implement, and demonstrate a secure Software Supply Chain framework within DevSecOps CI/CD pipelines by leveraging GitHub Actions, Snyk, CodeCloud, OWASP standards, ArgoCD, and Kubernetes, mitigating supply chain risks and improving software delivery security.

\section{Review of Related Work}
The topic of software supply chain security has gained significant attention due to high-profile attacks such as SolarWinds and Log4j. Current approaches, including vulnerability scanning tools like Snyk and static analysis platforms such as CodeCloud, focus on identifying known issues but often lack seamless integration into DevSecOps pipelines. OWASP provides a set of best practices for secure development, yet practical implementations vary widely across organizations. Tools like ArgoCD facilitate secure application delivery to Kubernetes clusters, but their use in holistic supply chain security strategies is underexplored.

While prior work has addressed individual aspects of pipeline security, such as static analysis or Kubernetes security, few studies have combined these elements into a cohesive, automated framework. This project seeks to build upon existing work by integrating these tools and methodologies into a streamlined DevSecOps pipeline, demonstrating practical improvements and scalability.

\section{Project Objective}
\begin{itemize}
    \item Develop a secure CI/CD pipeline using GitHub Actions.
    \item Integrate security tools such as Snyk and CodeCloud for automated scanning.
    \item Apply OWASP best practices to the pipeline.
    \item Deploy applications securely using ArgoCD and Kubernetes.
    \item Simulate and mitigate common supply chain attack scenarios.
    \item Document results, challenges, and recommendations.
\end{itemize}

\section{Description and Methodology of the Proposed Project}
This project aims to implement a secure Software Supply Chain framework by automating the integration of security tools and practices into a DevSecOps pipeline. The proposed methodology includes:

\subsection*{Pipeline Setup}
\begin{itemize}
    \item Create a basic CI/CD pipeline using GitHub Actions.
    \item Incorporate automated code and dependency scanning tools (Snyk, CodeCloud).
\end{itemize}

\subsection*{Security Enhancements}
\begin{itemize}
    \item Implement OWASP best practices for secure development.
    \item Set up role-based access control (RBAC) and network policies for Kubernetes.
\end{itemize}

\subsection*{Application Deployment}
\begin{itemize}
    \item Use ArgoCD for continuous delivery to Kubernetes.
    \item Ensure deployment security with signed container images and secure configurations.
\end{itemize}

\subsection*{Testing and Validation}
\begin{itemize}
    \item Simulate supply chain attacks (e.g., malicious dependencies).
    \item Assess pipeline resilience and effectiveness.
\end{itemize}

\subsection*{Documentation and Analysis}
\begin{itemize}
    \item Record methodology, results, challenges, and improvements.
\end{itemize}

\section{Resources}
\subsection*{Tools}
GitHub Actions, Snyk, CodeCloud, OWASP guidelines, ArgoCD.

\subsection*{Software}
Kubernetes (Minikube/AKS), Docker, secure image repositories.

\subsection*{Hardware}
Cloud-based or local testing environment (e.g., a VM).

\subsection*{Data}
Open-source vulnerable applications (e.g., Juice Shop).

\section{Contribution to Knowledge}
This project will contribute by:
\begin{itemize}
    \item Demonstrating a practical, secure DevSecOps pipeline framework.
    \item Highlighting the integration of diverse security tools and practices.
    \item Addressing gaps in the seamless automation of supply chain security.
    \item Providing recommendations for improving software delivery security.
\end{itemize}

While the project will focus on specific tools and scenarios, its methodology can be generalized for broader use cases. Foreseeable limitations include scalability to large, complex systems, which could be explored in future research.

\section{Project Schedule and Milestone Description}
\begin{itemize}
    \item \textbf{Week 1-2}: Research and define project scope; set up GitHub Actions.
    \item \textbf{Week 3-5}: Integrate Snyk and CodeCloud; document initial findings.
    \item \textbf{Week 6-8}: Apply OWASP practices; set up ArgoCD and Kubernetes.
    \item \textbf{Week 9-10}: Simulate and test supply chain attacks; refine pipeline security.
    \item \textbf{Week 11-12}: Document results; finalize paper and project presentation.
\end{itemize}

\section{References/Bibliography}
\begin{itemize}
    \item OWASP. "OWASP ZAP (Zed Attack Proxy)." \url{https://github.com/zaproxy/action-api-scan}
    \item Snyk. "Vulnerability Scanning for Developers." \url{https://snyk.io}
    \item CodeCloud. "Static and Dynamic Code Analysis Tools." \url{https://sonarcloud.io}
    \item ArgoCD. "GitOps Continuous Delivery." \url{https://argoproj.github.io}
    \item Kubernetes. "Official Documentation." \url{https://kubernetes.io/docs/home/}
    \item Github - Repository. "Official Repository of the Project." \url{https://github.com/BrenesRM/DevSecOps-NYIT-VAN}
\end{itemize}

\end{document}